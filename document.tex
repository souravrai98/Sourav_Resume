\documentclass[11pt,a4paper,sans]{moderncv} % Font sizes: 10, 11, or 12; paper sizes: a4paper, letterpaper, a5paper, legalpaper, executivepaper or landscape; font families: sans or roman

\moderncvstyle{casual} % CV theme - options include: 'casual' (default), 'classic', 'oldstyle' and 'banking'
\moderncvcolor{blue} % CV color - options include: 'blue' (default), 'orange', 'green', 'red', 'purple', 'grey' and 'black'

\usepackage{lipsum} % Used for inserting dummy 'Lorem ipsum' text into the template

\usepackage[scale=0.75]{geometry} % Reduce document margins
%\setlength{\hintscolumnwidth}{3cm} % Uncomment to change the width of the dates column
%\setlength{\makecvtitlenamewidth}{10cm} % For the 'classic' style, uncomment to adjust the width of the space allocated to your name
\usepackage[utf8]{inputenc}
%----------------------------------------------------------------------------------------
%	NAME AND CONTACT INFORMATION SECTION
%----------------------------------------------------------------------------------------

\firstname{Sourav} % Your first name
\familyname{Rai} % Your last name

% All information in this block is optional, comment out any lines you don't need
\title{Curriculum Vitae}

\photo[70pt][0.4pt]{pictures/picture} % The first bracket is the picture height, the second is the thickness of the frame around the picture (0pt for no frame)
%\quote{"A witty and playful quotation" - John Smith}

%----------------------------------------------------------------------------------------

\begin{document}




%----------------------------------------------------------------------------------------
%	CURRICULUM VITAE
%----------------------------------------------------------------------------------------

\makecvtitle % Print the CV title

%----------------------------------------------------------------------------------------
%	EDUCATION SECTION
%----------------------------------------------------------------------------------------

\section{Education}










\setlength{\arrayrulewidth}{1mm}
\setlength{\tabcolsep}{18pt}
\renewcommand{\arraystretch}{1.5}
\hspace*{-2.5cm}
\begin{tabular}{ |p{3cm}|p{3cm}|p{3cm}|p{3cm}|p{2cm}|  }
		
		
		\hline
		Degree & College/School & University & Passing Year & Pass Percentage \\
		\hline
		Class X & Kendriya Vidyalaya AFS Bamrauli Alld & CBSE & 2013 & 10 CGPA \\
		Class XII & Kendriya Vidyalaya AFS Bamrauli Alld   & CBSE & 2015 & 89.6\\
		B.Tech(IT) & KIET Group of Institutions Ghaziabad & AKTU & 2020 & 73.3(till 5th semester)\\
		
		\hline
\end{tabular}

\section{Projects}
\begin{enumerate}
	
	\item \textit{DRDO DRUSE PROJECT} {Developed code in python for the interaction of robot with the sensors and the raspberry pi}
	\newline{}
	\item Face Detection and Remote Surveillance using raspberry pi
	\newline{}
	\item Remotely Controlling Drone using ROS in eYRC - 2018
	\newline{}
\end{enumerate}




\section{Training and Internships}
	\cvitem{None}{}

\section{Research Publications}

	\cvitem{None}{}
	
\section{Technical Skills}
	
	\begin{itemize}
		\item Python
		\item C
		\item OpenCV
		\item ROS
		\item V-rep
		\item Linux
		\item Raspberry pi
		
	\end{itemize}
	


\end{document}